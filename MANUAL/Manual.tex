\documentclass[a4paper,12pt]{article}
\usepackage{fancyhdr}
\usepackage{fixltx2e}
\usepackage{amsmath}
\usepackage{graphicx}
\usepackage{braket}
\usepackage{amssymb}
\usepackage{mathtools}
\usepackage{mhchem}
\usepackage{wrapfig}
\usepackage{multirow}
\usepackage[export]{adjustbox}
\usepackage{titlesec}

\usepackage[usenames]{color}
\usepackage{hyperref}

\newcommand{\angstrom}{\mbox{\normalfont\AA}}
\DeclareSymbolFont{matha}{OML}{txmi}{m}{it}% txfonts
\DeclareMathSymbol{\varv}{\mathord}{matha}{118}
\setcounter{secnumdepth}{4}

\begin{document}
\numberwithin{equation}{section}
\numberwithin{figure}{section}
\pagestyle{fancy}



\section{Manual DSLEAP}
\subsection{Introduction}
This program collects algorithms for determining lattice oscillations.
The frequencies of lattice oscillations are important quantuties
in several areas of solid state physics.
Different approaches can be used to describe lattice oscillations, as
the dynamic structure factor (DSF), velocity autocorrelation function
(VACF), and projected velocity autocorrelations in $\mathbf{q}$-space.
Or the dynamic structure factor can be projected onto a set of
eigenvectors. The projected velocity autocorrelation (PVACF) is a projection
of the VACF onto a set of eigenvectors. The eigenvectors can for example
be taken from harmonic PhonoPy calculations.
For details about the implemented methods see \footnote[1]{Anharmonic lattice
dynamics in large thermodynamic ensembles with machine-learning force fields:
the breakdown of the phonon quasiparticle picture in $\textit{CsPbBr}_{3}$}




\subsection{Installation in UNIX}
To install the DSLEAP in UNIX unpack the tarball
tar -xvzf DSLEAP.tar.gz. Then go to the main folder
cd DSLEAP. Then export the desired gnu compiler
export FC=gfortran. Type cmake -DCMAKE\_BUILD\_TYPE=Release CMakeLists.txt.
Then type make and the executable will be found in ./bin/DSLEAP

\begin{table}[h!]
	\centering
	\caption{General MD spec flags}
\begin{tabular}{ |c|c| }
 \hline
	Step & command \\ 
 \hline
	(1) & tar -xvzf DSLEAP.tar.gz \\  
	(2) & cd DSLEAP \\
	(3) & export FC=gfortran \\
	(4) & cmake -DCMAKE\_BUILD\_TYPE=Release CMakeLists.txt \\
	(5) & make \\
	(6) & find executable in ./bin folder \\
 \hline
\end{tabular}
\end{table}




\subsection{General settings and the Phonon.in file}
There is a set of general settings for the phonon computation.
These flags will specify the conditions of the molecular dynamics (MD)
simulation which is analyzed.
The flags shown in table are specified in the \textbf{Phonon.in} file

\begin{table}[h!]
	\centering
	\caption{General MD spec flags}
\begin{tabular}{ |c|c|c|c| }
 \hline
	flag & meaning & type & default value \\ 
 \hline
	NSTART & structure in file to start sampling & integer & 1 \\  
	NEND   & total number of structures in XDATCAR file & integer & - \\
	DSTEP  & analyze every $DTSEP^{th}$ structure & integer & 1 \\
	TSTEP  & time step in what unit ever    & float & 1.0 \\
	NX     & number of periodic images in x & integer & - \\
	NY     & number of periodic images in y & integer & - \\
	NZ     & number of periodic images in z & integer & - \\
	DSTC   & time window for average correaltion functions & integer & NEND \\
	DYNSTRUC & True/False switch on/off Dynamic structure factor & bool & False\\
	PVACF   & on/off projected VACF on eigenvectors and $\mathbf{q}$-space & bool & False\\
	PVACK & on/off projected VACF $\mathbf{q}$-space only & bool & False\\
	PROJFAC & on/off projected DSF on eigenvectors & bool & False\\
 \hline
\end{tabular}
\end{table}

Specifing one of the methods with DSTC, DYNSTRUC, PVACF, PVACK or PROJFAC requires
different input files. The different methods and their usage will be described in the
following sections. The code also supports openmp parallelization. If you do not want
the program to use all your cores do export OMP\_NUM\_THREADS=\# of cores before execution.




\subsection{Dynamic structure factor computation}
To compute the DSF switch on the flag DYNSTRUC=True. And set the other remaining parameters
defined in the Phonon.in file. To compute the DSF you have to specify a $\mathbf{q}$-grid.
This can be done by supplying a $\mathbf{q}$-points file called "QVectors.in" containing on the first line
an integer with the number of $\mathbf{q}$-points. Then the $\mathbf{q}$-points are listed in
a 3 column format. If no "QVectors.in" file is supplied the default $\mathbf{q}$-points
of the commensurate $\mathbf{q}$-grid are used. As output you will obtain a file called
DynamicStructureFactor.out containing the DSF. The first column is the frequency axis. 
The frequency is inverse time units of the time step you supplied.
In the following columns the signals for the different $\mathbf{q}$-points are listed.
The file StructureFactor.out contains the static structure factor. The first column is
the Qpath then. And last the file StructureFactor\_vs\_t.out contains the structure factor in the
time domain. The first column of the file is the time axis then the signals for different 
$\mathbf{q}$-vectors follow.




\subsection{The velocity autocorrelation function in $\mathbf{q}$ space}
To compute the $\mathbf{q}$ space projection of the velocity autocorrelation function
set the flag PVACK=True. This method needs a masses.in file described
in section \ref{MassFile}. The masses are needed for weighing the velocities properly.
For this method the atoms have to be assigned to the unitcells
in which they are located. This can be done with the BoxList.in file described in
section \ref{BoxFile}. As output you are going to obtain numbered PVACF\_KO.out\{II\}
files where $II$ is an index from 1 up to the number of atoms per unit cell.
The first column of this file contains the frequency grid and the following columns
contain the signals for the used $\mathbf{q}$-vectors.
The frequency grid is given in inverse time units as
defined in the Phonon.in file. In the files PVACF\_KO\_vs\_t.out\{II\} you will find
the time signals. These have the time axis on the first column and then
the signals for the different $\mathbf{q}$-values follow.




\subsection{Projected velocity autocorrelation PVACF}
To compute the projected velocity autocorrelation function
set the flag\\ PVACF=True. This computes a PVACF projected onto
a set of basis vectors and onto a set of $\mathbf{q}$-vectors.
The routine needs the input files BoxList.in, masses.in and
BasisVector.in files. If no BasisVector.in file is supplied
the code will try to find a QVectors.in file. If this file
is found the $\mathbf{q}$-vectors from this file are used
and a diagonal basis. If no QVectors.in file
is supplied the default $\mathbf{q}$-vectors will be used
in combination with the diagonal basis.
The masses.in file is needed to weigh the velocities with their
atomic masses and the BoxList.in is needed for the spatial
fourier transforms.
As output you will obtain PVACF.out\{II\} files. These 
files will contain the frequency axis on the first column and
the signals for every $\mathbf{q}$-vector on the following columns.
Such a file will be written for every phonon branch denoted by the counter
\{II\}. The files PVACF\_vs\_t.out\{II\} will contain the corresponding
time signals with the time axis on the first column. Then the signals
for the used $\mathbf{q}$-points are listed.




\subsection{Projected Dynamic Structure factor}
To compute the projected dynamic structure factor
set the flag PROJFAC=True. This computes a DSF projected onto
a set of basis vectors and onto a set of $\mathbf{q}$-vectors.
The routine needs the input files BoxList.in and
BasisVector.in. If no BasisVector.in file is supplied
the code will try to find a QVectors.in file. If this file
is found the $\mathbf{q}$-vectors from this file are used
and a diagonal basis. If no QVectors.in file
is supplied the default $\mathbf{q}$-vectors will be used
in combination with the diagonal basis.
As output you will obtain ProjectedDSF.out\{II\} files. These 
files will contain the frequency axis on the first column and
the signals for every $\mathbf{q}$-vector on the following columns.
Such a file will be written for every phonon branch numbered by \{II\}.
The files StructureFactorProj\_vs\_t.out\{II\} will contain the corresponding
time signals with the time axis on the first column. Then the signals
for the used $\mathbf{q}$-points are listed.




\subsection{Input file \textit{BoxList.in}}\label{BoxFile}
The BoxList.in file contains a connection table assigning every atom
to one of the unit cells building up the super cell.
The file contains a table where the number of lines are the number of unitcells
building up the supercell. Every line contains N integer numbers, where N defines
the number of atoms per unitcell. The numbers represent the atom number in the 
XDATCAR file. This file is needed for PVACF and PVACK.




\subsection{Input file \textbf{masses.in}}\label{MassFile}
The masses.in contains the number of atoms in the unit cell on the first line.
The following lines contain a single number describing the mass of the atoms in the unit cell.
The order has to be the same as in the XDATCAR file.
This file is needed for PVACF and PVACK.




\subsection{The \textbf{BasisVector.in} file}
This file is needed when computing PVACF or PROJFAC. The projected velocity autocorrelation
function in $\mathbf{q}$-space and onto a set of basis vectors. The PROJFAC
is a projection of the dynamic structure factor onto the same set of eigenvectors.
The file contains 4 integer numbers on the first line. The first number is the 
number of different $\mathbf{q}$-points. The second number is the number of branches per
$\mathbf{q}$-point. The third number defines the number of atoms in the unit cell. And the
last number defines the dimensionality of the underlying space, which is always 3.
The file is then structured as follows. The next line contains the first $\mathbf{q}$-vector.
Then without an empty line in-between the first phonon branch of $\mathbf{q}$-vector one follows.
Then an empty line and the next phonon branch follows. The number of lines per
branch is the number of atoms per unit cell.
Like this all branches for the first $\mathbf{q}$ point are listed. Then 2 empty lines
follow and the next $\mathbf{q}$-vector is defined. Followed by the first phonon branch of
the second $\mathbf{q}$-point.
In the folder toolsPy/MakeBasisVectorIn.py you can find a script to extract eigenvectors
from a phonopy band.yaml file. The file contains a routine LoadEigenVectors.WriteOutput 
that can be adapted to write your own Eigenvectors in the proper format.



\subsection{Example with input files}
The folder Example shows representatives for all the input files needed to do a full analysis.
The example treats a cubic $16\times16\times16$ 'Morse-crystal' at $100$~K with a single atom
in the unitcell. Since a trajectory file would be too large to supply, the output files
from the various methods are put there to take a look at.
There is also a py script MakeBoxList.py that generates the \textit{BoxList.in} file for
this specific example. And there is another py script MakeEigenVectors.py generating the
BasisVector.in file for this example.









\end{document}
